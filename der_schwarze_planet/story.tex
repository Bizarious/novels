\documentclass[a4paper,11pt]{book}
\usepackage[ngerman]{babel}

\title{Der schwarze Planet}

\begin{document}
\maketitle

\chapter{Prolog}
\textit{2.004.298 v.Chr. nach irdischer Zeitrechung}
\\

Die Sonne stand bereits an ihrem höchsten Punkt, und dennoch war der Boden so schwach beleuchtet, dass Menschen Schwierigkeiten hätten, überhaupt etwas zu erkennen. Die dichte, pechschwarze Vegetation gab sich größte Mühe, jedes bisschen des tiefroten Lichts zu absorbieren und nur wenige Strahlen schafften es überhaupt, die fast unüberwindliche Hürde aus Blättern und Geäst zu durchdringen. Doch mitten in diesen ewigen Schatten, für das menschliche Auge kaum wahrnehmbar, lugten zwischen dem (trotz der mangelnden Beleuchtung) dichten Bodenbewuches zwei Augen hervor.

Eine kleine, reptilienartige Kreatur, nicht nennenswert größer als ein Schimpanze von der Erde, saß ängstlich und wie angewurzelt an Ort und Stelle. Mit ihren extrem lichtempfindlichen Augen tastete sie die Umgebung akribisch nach jedweder Form der Gefahr ab. Ihr war vollkommen bewusst, dass jedes Anzeichen von Bewegung und Geräusch innerhalb von Sekunden zu ihrem Tod führen konnten. Eine halbe Ewigkeit schon verharrte sie in dieser Position. Noch wenige Stunden zuvor war die Kreatur mit einer Gruppe von Artgenossen unterwegs zur Nahrungssuche, bevor sie hinterrücks attackiert wurden. Der Angreifer, ein etwa 6 Mal so großes Raubtier mit nadelartig spitzen und einen halben Meter langen Zähnen, stürzte sich aus dem Hinterhalt heraus auf die nichts ahnende Gruppe. Trotz seiner Größe war es zu erstaunlich leiser und leichtgängiger Fortbewegung im Stande, wodurch es selbst durch äußerste Aufmerksamkeit nur schwer wahrgenommen werden konnte. Sein komplett schwarzes Schuppenkleid machte es zudem nahezu unsichtbar gegenüber der gleichfarbigen Vegetation. Die kleine Kreatur war noch jung und kannte diesen Räuber bislang nur von den Erzählungen ihrer Artgenossen. Sie fragte sich schon immer, wie ein so riesiges Tier solche Fähigkeiten überhaupt besitzen konnte.

\end{document}